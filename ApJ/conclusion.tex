\section{Conclusion}\label{sec:conclusion}
It is, at this point, well established that the Jao Gap may provide a unique view of the interiors of stars for which other probes, such as seismology, fail. However, it has only recently become clear that the Gap may lend insight into not just structural changes within a star but also into the magnetic environment of the star.
\citet{Jao2023} presented evidence that the physics driving the Gap might additionally result in a paucity of H$\alpha$ emission. These authors propose potential physical mechanisms which could explain this paucity, including the core of the star acting as an angular momentum sink during mixing events.

Here we have expanded upon this work by probing the degree and variability of
Calcium II H\&K emission around the Jao Gap. We lack the same statistical
power of \citeauthor{Jao2023}'s sample; however, by focusing on the
star-to-star variability within magnitude bins we are able to retain
statistical power. We find that there is an anomalous increase in variability
at a G magnitude of $\sim 11$. This is only slightly below the observed mean gap magnitude.

Additionally, we propose a simple model to explain this variability. Making the
assumption that the periodic convective mixing events will have some small but
random effect on the overall magnetic field strength we are able to
qualitatively reproduce the increase activity spread in a synthetic population
of stars. 
