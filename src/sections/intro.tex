\section{INTRODUCTION}\label{sec:intro}
Due to the initial mass requirements of the molecular clouds which collapse to form
stars, star formation is strongly biased towards lower mass, later spectral
class stars when compared to higher mass stars. Partly as a result of this
bias and partly as a result of their extremely long main-sequence lifetimes,
M Dwarfs make up approximately 70 percent of all stars in the galaxy. Moreover,
some planet search campaigns have focused on M Dwarfs due to the relative ease
of detecting small planets in their habitable zones \citep[e.g.][]{Nut08}.
M Dwarfs then represent both a key component of the galactic stellar population
as well as the most numerous possible set of stars which may host habitable exoplanets.
Given this key location M Dwarfs occupy in modern astronomy it is important to
have a thorough understanding of their structure and evolution.

\citet{Jao2018} discovered a novel feature in the Gaia Data Release 2 (DR2)
$G_{BP}-G_{RP}$ color-magnitude-diagram. Around $M_{G}=10$ there is an
approximately 17 percent decrease in stellar density of the sample of stars
\citet{Jao2018} considered. Subsequently, this has become known as either the
Jao Gap, or Gaia M Dwarf Gap. Following the initial detection of the Gap in DR2
the Gap has also potentially been observed in 2MASS \citep{Skrutskie2006,
Jao2018}; however, the significance of this detection is quite weak and it
relies on the prior of the Gap's location from Gaia data. The Gap is
also present in Gaia Early Data Release 3 (EDR3) \citep{Jao2021}. These EDR3
and 2MASS data sets then indicate that this feature is not a bias inherent to
DR2.

The Gap is generally attributed to convective instabilities in the cores of
stars straddling the fully convective transition mass (0.3 - 0.35 M$_{\odot}$)
\citep{Baraffe2018}. These instabilities interrupt the normal, slow, main
sequence luminosity evolution of a star and result in luminosities lower
than expected from the main sequence mass-luminosity relation \citep{Jao2020}.

The Jao Gap, inherently a feature of M Dwarf populations, provides an enticing
and unique view into the interior physics of these stars \citep{Feiden2021}.
This is especially important as, unlike more massive stars, M Dwarf seismology
is infeasible due to the short periods and extremely small
magnitudes which both radial and low-order low-degree non-radial seismic waves
are predicted to have in such low mass stars \citep{Rodriguez-Lopez2019}. The
Jao Gap therefore provides one of the only current methods to probe the
interior physics of M Dwarfs.

The magnetic activity of M dwarfs is of particular interest due to the
theorised links between habitability and the magnetic environment which a
planet resides within. M dwarfs are known to be more magnetically active than
earlier type stars while simultaneously this same high activity calls into
question the canonical magnetic dynamo believed to drive the magnetic field of
solar like stars (the $\alpha\Omega$ dynamo). One primary challenge which M
dwarfs pose is that stars less than approximately 0.35 M$_{\odot}$ are composed
of a single convective region. This denies any dynamo model differential
rotation between adjacent levels within the star. Alternative dynamo models
have been proposed, such as the $\alpha^{2}$ dynamo along with modifications to
the $\alpha\Omega$ dynamo which may be predictive of M dwarf magnetic fields.

Despite this work, very few studies have dived specifically into the magnetic
field of M dwarfs at or near the convective transition region
{\color{red}(CITATION)}. This is not surprising as that only spans
approximately 0.2 mag in the Gaia BP-RP color magnitude diagram and is
therefore populated by a relatively small number of stars. 

\citep{Jao2023} identify the Jao gap as a strong discontinuity point for
magnetic activity in M dwarfs. Two primary observations from their work are
that the Gap serves as a boundary where very few active stars in their sample
of 640 M dwarfs exist below the gap and that the overall downward trend of
activity moving to fainter magnitudes is anomalously high in within the 0.2 mag
range of the gap. \citeauthor{Jao2023} Figures 3 and 13 make this paucity in
H$\alpha$ emission particularly clear. Based on previous work from
{\color{red}Spada \& Lanzafam 2020}, {\color{red}Curtis et al. 2020, and Dungee
et al. 2022} the authors propose that the mechanism resulting in the reduced
fraction of active stars within the gap is that as the radiative zone dissipates
due to core expansion, angular momentum from the outer convective zone is
dumped into the core resulting in a faster spin down than would otherwise be
possible. Effectively the core of the star acts as a sink, reducing the amount
of angular momentum which needs to be lost by magnetic breaking for the outer
convective region to reach the same angular velocity. Given that H$\alpha$
emission is strongly coupled magnetic activity in the lower photosphere and
that a stars angular velocity is a primary factor in its magnetic activity, a
faster spin down will serve to more quickly dampen H$\alpha$ activity.

In addition to H$\alpha$ the Calcium Fraunhaufer lines may be used to trace the
magnetic activity of a star. These lines originate from magnetic heating of the
upper chromosphere driven by magnetic shear stresses within the star.
\citet{Boudreaux2022} and \citet{Perdelwitz2021} presented calcium emission
measurements for stars spanning the Jao Gap. In this paper we search for similar
trends in the Ca II H\& K emission as \citeauthor{Jao2023} see in the H$\alpha$
emission. In Section \ref{sec:results} we investigate the empirical
star-to-star variability in emission and quantify if this could be due to
noise or sample bias; in Section \ref{sec:modeling} we present a simplified toy
model which shows that the mixing events characteristic of convective kissing
instabilities could lead to increased star-to-star variability in activity as
is seen empirically.

% Stellar modeling has been successful in reproducing the Jao Gap
% \citep[e.g.][]{Feiden2021,Mansfield2021} and, with these models, we have begun
% to understand which parameters constrain the Jao Gap's location. For example,
% it is now well documented that metallicity affects the Jao Gap's color, with
% higher metallicity stellar populations showing the Jao Gap at consistently
% higher masses / bluer colors \citep{Mansfield2021}.
%
% {\color{red} EXPAND THIS, READ SOME OTHER GAP PAPERS TO SEE WHAT THEY DO}

% Both \citeauthor{Feiden2021} and \citeauthor{Mansfield2021} demonstrate the Jao
% Gap's location sensitivity to age, evolving to higher mass regions of the
% mass-luminosity relation with population age. Per \citet{Mansfield2021} the
% degree of this location evolution also does not seem to be strongly sensitive
% to metallicity. 
