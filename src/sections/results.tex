\section{Correlation}\label{sec:results}
Using Ca II H\&K emission data from \citet{Boudreaux2021} and
\citet{Perdelwitz2021} (quantified using the $R_{HK}$ metric) we investigate
the correlation between the Jao Gap magnitude and stellar magnetic activity. We
are more statistically limited here than past authors have been due to
therequirment for high resolution spectoscopic data when measuring Calcium
emission; however, this is balanced by the aparent stronger correlation between
Cacium emission and the Jao gap when compared to H$\alpha$ emission. 

The merged dataset is presented in Figure \ref{fig:mergedData}. There is a
visual discontinuity around the Jao Gap mangitude; however, this manifests as
an increase in the spread of the emission measurments rather than a change in
the mean value. In order to quantify the signifigance of this discontinuity we
measure the false alarm probaility of the change in standard deviation.

\begin{figure}
  \centering
  \includegraphics[width=0.45\textwidth]{figures/Combined.pdf}
  \caption{Merged Dataset from \citet{Boudreaux2021, Perdelwitz2021}. Note the
  increase in the spread of $R'_{HK}$ around the Jao Gap Magnitude.}
  \label{fig:mergedData}
\end{figure}

First bin the merged dataset into bins with a width of 0.5 mag. In each bin we
measure the standard deviation about the mean of the data. The results of this
are shown in Figure \ref{fig:deviation}. In order to measure the false alarm
probabililty of this discontinuity we first resample the merged calcium
emission data based on the associated uncertanties for each datapoint as
presented in their respective publications. Then, for each of these ``resample
trials'' we measure the probility that a change in the standard deviation of
the size seen would happen purley due to noise. Results of this test are show in
in Figure \ref{fig:dist}. 

\begin{figure}
  \centering
  \includegraphics[width=0.45\textwidth]{figures/Deviation.pdf}
  \caption{Standard deviation of Calcium emission data within each bin. Note
  the discontinuity near the Jao Gap Magnitude.}
  \label{fig:deviation}
\end{figure}

\begin{figure}
  \centering
  \includegraphics[width=0.45\textwidth]{figures/fpDist.pdf}
  \caption{Probability distribution of the false alarm probability for the
  discontinuity seen in Figure \ref{fig:deviation}. The mean of this
  distribution is $0.341\%\pm^{0.08}_{0.08}$.}
  \label{fig:dist}
\end{figure}
