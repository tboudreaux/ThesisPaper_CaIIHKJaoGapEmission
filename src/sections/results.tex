\section{Correlation}\label{sec:results}
Using Ca II H\&K emission data from \citet{Boudreaux2022} and
\citet{Perdelwitz2021} (quantified using the $R_{HK}$ metric) we investigate
the correlation between the Jao Gap magnitude and stellar magnetic activity. We
are more statistically limited here than past authors have been due to
therequirment for high resolution spectoscopic data when measuring Calcium
emission; however, this is balanced by the aparent stronger correlation between
Cacium emission and the Jao gap when compared to H$\alpha$ emission. 

The merged dataset is presented in Figure \ref{fig:mergedData}. There is a
visual discontinuity around the Jao Gap mangitude; however, this manifests as
an increase in the spread of the emission measurments rather than a change in
the mean value. In order to quantify the signifigance of this discontinuity we
measure the false alarm probaility of the change in standard deviation.

\begin{figure}
  \centering
  \includegraphics[width=0.45\textwidth]{figures/Combined.pdf}
  \caption{Merged Dataset from \citet{Boudreaux2022, Perdelwitz2021}. Note the
  increase in the spread of $R'_{HK}$ around the Jao Gap Magnitude.}
  \label{fig:mergedData}
\end{figure}

First bin the merged dataset into bins with a width of 0.5 mag. In each bin we
measure the standard deviation about the mean of the data. The results of this
are shown in Figure \ref{fig:deviation}. In order to measure the false alarm
probabililty of this discontinuity we first resample the merged calcium
emission data based on the associated uncertanties for each datapoint as
presented in their respective publications. Then, for each of these ``resample
trials'' we measure the probility that a change in the standard deviation of
the size seen would happen purley due to noise. Results of this test are show in
in Figure \ref{fig:dist}. 

\begin{figure}
  \centering
  \includegraphics[width=0.45\textwidth]{figures/Deviation.pdf}
  \caption{Standard deviation of Calcium emission data within each bin. Note
  the discontinuity near the Jao Gap Magnitude.}
  \label{fig:deviation}
\end{figure}

\begin{figure}
  \centering
  \includegraphics[width=0.45\textwidth]{figures/fpDist.pdf}
  \caption{Probability distribution of the false alarm probability for the
  discontinuity seen in Figure \ref{fig:deviation}. The mean of this
  distribution is $0.341\%\pm^{0.08}_{0.08}$.}
  \label{fig:dist}
\end{figure}

This rapid increase star-to-star variability would only arise due purley to
noise $0.3\pm0.08$ percent of the time and is therefore likeley either a true
effect or an alias of some sample bias. {\color{red} COME BACK TO HERE TO FLUSH
OUT SAMPLE BIAS SECTION.}

If the observed increase in variability is not due to a sample bias and rather
is a physically driven effect then there is an obvious similarity between these
findings and those of \citep{Jao2023}. Specifically we find a increase in
variability just below the magnitude of the gap. Moreover, this variability
increase is primarily driven by an increase in the number of low activity stars
(as opposed to an increase in the number of high activity stars). We can
further investigate the observed change in variability for only low activity
stars by filtering out those stars at or above the saturated threshold for
magnetic activity. \citet{Boudreaux2022} identify $\log(R'_{HK}) = -4.436$ as
the saturation threshold. We adopt this value and filter out all stars where
$\log(R'_{HK}) \geq -4.436$. Applying the same analysis to this reduced dataset
as was done to the full dataset we still find a discontinuity at the same
location (Figure \ref{fig:reduced}). This discontinuity is of a smaller
magnitude and consequently is more likley to be due purley to nouse, with a
$7\pm0.2$ percent false alarm probability. This false alarm probility is
however only concerned with the first point after the jump in variability. If
we consider the false alarm probability of the entire high variabilty region
then the probability that the high variability region is due purley to noise
drops to $1.4\pm0.04$ percent.

\begin{figure}
  \centering
  \includegraphics[width=0.45\textwidth]{figures/ReducedDeviation.pdf}
  \caption{Spread in the magnetic activity metric for the merged sample with
  any stars $\log(R'_{HK}) > -4.436$ filterd out.}
  \label{fig:reduced}
\end{figure}

We observe a strong, likeley statistically signifigant, discontinuty in the
star-to-star variability of Ca II K \& K emission coincident with the magnitude
of the Jao Gap. However, modeling is required to determine if this discontinuty
may be due to the same underlying physics.
