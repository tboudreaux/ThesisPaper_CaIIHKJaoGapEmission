\documentclass[twocolumn]{aastex62}
\newcommand{\vdag}{(v)^\dagger}
\newcommand\aastex{AAS\TeX}
\newcommand\latex{La\TeX}

\usepackage{amsmath}
\usepackage{cancel}
\usepackage{float}

\shorttitle{Updated Opacities for DSEP}
\shortauthors{Boudreaux et al.}
% \watermark{DRAFT}
% \graphicspath{{./}{figures/}{src/figures}}

\begin{document}

\title{Correlations between Ca II H\&K Emission and the Gaia M dwarf Gap}

\correspondingauthor{Emily M. Boudreaux}
\email{thomas.m.boudreaux.gr@dartmouth.edu,\\thomas@boudreauxmail.com}

\author[0000-0002-2600-7513]{Emily M. Boudreaux}
\affiliation{Department of Physics and Astronomy, Dartmouth College, Hanover, NH 03755, USA}

\author[0000-0003-3096-4161]{Brian C. Chaboyer}
\affiliation{Department of Physics and Astronomy, Dartmouth College, Hanover, NH 03755, USA}

\received{}
\revised{}
\revised{}
\accepted{}
\submitjournal{ApJ}


\begin{abstract}
Previous work has demonstrated a paucity of Hydrogen Alpha emission at the same
  Gaia G Magnitude as the Jao Gap in the solar neighborhood. The exact
  mechanism which results in this paucity is as yet unknown; however, the
  authors of the originating paper suggestion that it may be the result of
  complex variations to a stars magnetic topology driven by the Jao Gaps
  characteristic formation and breakdown of stars radiative transition zones.
  Here I present a brief summary of a potential extension to this work looking
  at Ca II H\&K emission lines. Preliminary work with archival data shows a
  much stronger correlation between the calcium emission lines and the Jao Gap
  than was observed between hydrogen emission and the Jao Gap. If this
  observation withstands further testing then it may provide a new way to
  locate the gap in populations which lack the large counting statistics
  currently required --- and which are only practically available from Gaia
  data at moment.

\end{abstract}

\keywords{Stellar Evolution (1599) --- Stellar Evolutionary Models (2046)}

\section{INTRODUCTION}\label{sec:intro}
Due to the initial mass requirements of the molecular clouds which collapse to form
stars, star formation is strongly biased towards lower mass, later spectral
class stars when compared to higher mass stars. Partly as a result of this
bias and partly as a result of their extremely long main-sequence lifetimes,
M Dwarfs make up approximately 70 percent of all stars in the galaxy. Moreover,
some planet search campaigns have focused on M Dwarfs due to the relative ease
of detecting small planets in their habitable zones \citep[e.g.][]{Nut08}.
M Dwarfs then represent both a key component of the galactic stellar population
as well as the possible set of stars which may host habitable exoplanets.
Given this key location M Dwarfs occupy in modern astronomy it is important to
have a thorough understanding of their structure and evolution.

\citet{Jao2018} discovered a novel feature in the Gaia Data Release 2 (DR2)
$G_{BP}-G_{RP}$ color-magnitude-diagram. Around $M_{G}=10$ there is an
approximately 17 percent decrease in stellar density of the sample of stars
\citet{Jao2018} considered. Subsequently, this has become known as either the
Jao Gap, or Gaia M Dwarf Gap. Following the initial detection of the Gap in DR2
the Gap has also potentially been observed in 2MASS \citep{Skrutskie2006,
Jao2018}; however, the significance of this detection is quite weak and it
relies on the prior of the Gap's location from Gaia data. Further, the Gap is
also present in Gaia Early Data Release 3 (EDR3) \citep{Jao2021}. These EDR3
and 2MASS data sets then indicate that this feature is not a bias inherent to
DR2.

The Gap is generally attributed to convective instabilities in the cores of
stars straddling the fully convective transition mass (0.3 - 0.35 M$_{\odot}$)
\citep{Baraffe2018}. These instabilities interrupt the normal, slow, main
sequence luminosity evolution of a star and result in luminosities lower
than expected from the main sequence mass-luminosity relation \citep{Jao2020}.

The Jao Gap, inherently a feature of M Dwarf populations, provides an enticing
and unique view into the interior physics of these stars \citep{Feiden2021}.
This is especially important as, unlike more massive stars, M Dwarf seismology
is infeasible due to the short periods and extremely small
magnitudes which both radial and low-order low-degree non-radial seismic waves
are predicted to have in such low mass stars \citep{Rodriguez-Lopez2019}. The
Jao Gap therefore provides one of the only current methods to probe the
interior physics of M Dwarfs.

\citep{Jao2023} identify the Jao gap as a strong discontinuity point for
activity in M dwarfs. Two primary observations from their work are that the Gap
serves as a boundary where very few active stars in their sample of 640 M
dwarfs exist below the gap and that the overall downward trend of activity
moving to fainter magnitudes is anomalously high in within the 0.2 mag range of
the gap. Their figures 3 and 13 are of particular relevance here and have been
included below for convince. Based on previous work from {\color{red}Spada \& Lanzafam 2020},
{\color{red}Curtis et al. 2020, and Dungee et al. 2022} the authors propose that the
mechanism resulting in the reduced fraction of active stars within the gap is
that as the radiaive zone disipates due to core expansion, angular momentum
from the outter convective zone is dumped into the core resulting in a faster
spin down than would otherwise be possible. Effectively the core of the star
acts as a sink, reducing the amount of angular momentum which needs to be lost by
magnetic breaking for the outer convective region to reach the same angular
velocity. Given that H$\alpha$ emission is strongly coupled magnetic activity in the
lower photosphere and that a stars angular velocity is a primary factor in its
magnetic activity, a faster spin down will serve to more quickly dampen H$\alpha$
activity.


% Stellar modeling has been successful in reproducing the Jao Gap
% \citep[e.g.][]{Feiden2021,Mansfield2021} and, with these models, we have begun
% to understand which parameters constrain the Jao Gap's location. For example,
% it is now well documented that metallicity affects the Jao Gap's color, with
% higher metallicity stellar populations showing the Jao Gap at consistently
% higher masses / bluer colors \citep{Mansfield2021}.
%
% {\color{red} EXPAND THIS, READ SOME OTHER GAP PAPERS TO SEE WHAT THEY DO}

% Both \citeauthor{Feiden2021} and \citeauthor{Mansfield2021} demonstrate the Jao
% Gap's location sensitivity to age, evolving to higher mass regions of the
% mass-luminosity relation with population age. Per \citet{Mansfield2021} the
% degree of this location evolution also does not seem to be strongly sensitive
% to metallicity. 


\section{Correlation}\label{sec:results}
Using Ca II H\&K emission data from \citet{Boudreaux2022} and
\citet{Perdelwitz2021} (quantified using the $R_{HK}$ metric) we investigate
the correlation between the Jao Gap magnitude and stellar magnetic activity. We
are more statistically limited here than past authors have been due to
the requirement for high resolution spectroscopic data when measuring Calcium
emission; however, this is balanced by the apparent stronger correlation between
Calcium emission and the Jao gap when compared to H$\alpha$ emission. 

The merged dataset is presented in Figure \ref{fig:mergedData}. There is a
visual discontinuity just below the Jao Gap magnitude; however, this
manifests as an increase in the spread of the emission measurements rather than
a change in the mean value. In order to quantify the significance of this
discontinuity we measure the false alarm probability of the change in standard
deviation.

\begin{figure}
  \centering
  \includegraphics[width=0.45\textwidth]{figures/Combined.pdf}
  \caption{Merged Dataset from \citet{Boudreaux2022, Perdelwitz2021}. Note the
  increase in the spread of $R'_{HK}$ around the Jao Gap Magnitude.}
  \label{fig:mergedData}
\end{figure}

First we split the merged dataset into bins with a width of 0.5 mag. In each bin we
measure the standard deviation about the mean of the data. The results of this
are shown in Figure \ref{fig:deviation}. In order to measure the false alarm
probability of this discontinuity we first resample the merged calcium
emission data based on the associated uncertainties for each datum as
presented in their respective publications. Then, for each of these ``resample
trials'' we measure the probability that a change in the standard deviation of
the size seen would happen purely due to noise. Results of this test are show in
in Figure \ref{fig:dist}. 

\begin{figure}
  \centering
  \includegraphics[width=0.45\textwidth]{figures/Deviation.pdf}
  \caption{Standard deviation of Calcium emission data within each bin. Note
  the discontinuity near the Jao Gap Magnitude.}
  \label{fig:deviation}
\end{figure}

\begin{figure}
  \centering
  \includegraphics[width=0.45\textwidth]{figures/fpDist.pdf}
  \caption{Probability distribution of the false alarm probability for the
  discontinuity seen in Figure \ref{fig:deviation}. The mean of this
  distribution is $0.341\%\pm^{0.08}_{0.08}$.}
  \label{fig:dist}
\end{figure}

This rapid increase star-to-star variability would only arise due purely to
noise $0.3\pm0.08$ percent of the time and is therefore likely either a true
effect or an alias of some sample bias. {\color{red} COME BACK TO HERE TO FLUSH
OUT SAMPLE BIAS SECTION.}

If the observed increase in variability is not due to a sample bias and rather
is a physically driven effect then there is an obvious similarity between these
findings and those of \citep{Jao2023}. Specifically we find a increase in
variability just below the magnitude of the gap. Moreover, this variability
increase is primarily driven by an increase in the number of low activity stars
(as opposed to an increase in the number of high activity stars). We can
further investigate the observed change in variability for only low activity
stars by filtering out those stars at or above the saturated threshold for
magnetic activity. \citet{Boudreaux2022} identify $\log(R'_{HK}) = -4.436$ as
the saturation threshold. We adopt this value and filter out all stars where
$\log(R'_{HK}) \geq -4.436$. Applying the same analysis to this reduced dataset
as was done to the full dataset we still find a discontinuity at the same
location (Figure \ref{fig:reduced}). This discontinuity is of a smaller
magnitude and consequently is more likely to be due purely to noise, with a
$7\pm0.2$ percent false alarm probability. This false alarm probability is
however only concerned with the first point after the jump in variability. If
we consider the false alarm probability of the entire high variability region
then the probability that the high variability region is due purely to noise
drops to $1.4\pm0.04$ percent.

\begin{figure}
  \centering
  \includegraphics[width=0.45\textwidth]{figures/ReducedDeviation.pdf}
  \caption{Spread in the magnetic activity metric for the merged sample with
  any stars $\log(R'_{HK}) > -4.436$ filtered out.}
  \label{fig:reduced}
\end{figure}

We observe a strong, likely statistically significant, discontinuity in the
star-to-star variability of Ca II K \& K emission just below the magnitude
of the Jao Gap. However, modeling is required to determine if this discontinuity
may be due to the same underlying physics.

While the observed increase in variability seen here does not seem to be
coincident with the Jao Gap --- instead appearing to be approximately 0.5 mag
fainter, in agreement with what is observed in \citet{Jao2023} --- a number of
complicating factors prevent us from falsifying that the these two features are
not coincident. \citeauthor{Jao2023} find, similar to the results presented
here, that the paucity of $H\alpha$ emission originates just below the gap.
Moreover, we use a 0.5 magnitude bin size when measuring the star-to-star
variability which injects error into the positioning of any feature in
magnitude space. We can quantify the degree of uncertainty the magnitude bin
choice injects by conducting Monte Carlo trials where bins are randomly shifted
redder or bluer. We conduct 10,000 trials where each trial involves sampling a
random shift to the bin start location from a normal distribution with a
standard deviation of 1 magnitude. For each trial we identify the discontinuity
location as the maximum value of the gradient of the standard deviation
(this is the derivative of the data in Figure \ref{fig:reduced}). Some trials
result in the maximal value lying at the 0th index of the magnitude array due
to edge effects, these trials are rejected (and account for 11\% of the
trials). The uncertainty in the identified magnitude of the discontinuity due
to the selected start point of the magnitude bins reveals a $1\sigma = \pm$0.32
magnitude uncertainty in the location of the discontinuity (Figure
\ref{fig:GapLocationMC}). Finally, all previous studies of the M dwarf gap
\citep{Jao2018, Feiden2022, Mansfield2021, Boudreaux2022, Jao2023} demonstrate
that the gap has a color dependency, shifting to fainter magnitudes as the
population reddens and consequently an exact magnitude range is ill-defined.
Therefore we cannot falsify the model that the discontinuity in star-to-star activity
variability is coincident with the Jao Gap magnitude.

\begin{figure}
  \centering
  \includegraphics[width=0.45\textwidth]{figures/GapLocationMC.pdf}
  \caption{Probability density distribution of discontinuity location as
  identified in the merged dataset. The dashed line represents the mean of the
  distribution while the shaded region runs from the 16th percentile to the
  84th percentile of the distribution. This distribution was built from 10,000
  independent samples where the discontinuity was identified as the highest
  value in the gradient of the standard deviation.}
  \label{fig:GapLocationMC}
\end{figure}

\subsection{Rotation}
It is well known that stars magnetic activity tend to be correlated with their
rotational velocity {\color{red}[CITATIONS]}; therefore, we investigate whether
there is a similar correlation between gap location and rotational period in
our dataset. All targets from \citet{Boudreaux2022} already have published
rotational periods; however, targets from \citet{Perdelwitz2021} do not
nessisarily have published periods. Therefore, we derive photometric rotational
periods for these targets here. Given the inherent heterogeneity of M Dwarf
stellar surfaces {\color{red}[CITATION]} we are able to determine the
rotational period of a star through the analysis of active regions. Various
methodologies can be employed for this purpose, including the examination of
photometry and light curves \citep[e.g.,][]{Newton2016}, and the observation of
temporal changes in the strength of chromospheric emission lines such as Ca II
H\&K or $H\alpha$ \citep[e.g.,][]{2019A&A...623A..24F,2023MNRAS.518.3147K}. New
rotatioal periods are derived from TESS 2-minute cadence data. \footnote{Some M
Dwarfs lacking a documented rotational period did not have sufficient TESS data
to yield fiducial rotational periods}.

Due to both the large frequency and amplitudes of M dwarf flaring rates the
photometic period can proove difficulut to measure --- as frequency directly
correlates with periodicity. Thus, following the process described in
\citet{2023AJ....165..192G}, we utilize two methods in this paper to reduce the
affect of flares. One method uses \texttt{stella} {\color{red}[CITATION]} a
python package which impliments an series of pre-trained convolutional neural
networks (CNNs) to remove flare-shaped features in a light curve. The second
method seperates a stars photometry into 10 minute bins to account for
misshapen flares which the CNN \texttt{stella} is known to be biassed against.

\texttt{stella} \citep{FeinsteinFlare2020,FeinsteinStella2020} employs a
diverse library of models trained with varying initial seeds. The Convolutional
Neural Networks in \texttt{stella} are trained on labeled TESS 2-min for both
flares and non-flares. For the purposes of this paper, we use an ensemble of
100 models in \texttt{stella}'s library to optimize the gains
{\color{green}[Emily: Aylin, what are gains?]}. \texttt{stella} scores flairs
with a probability of between 0 to 1 --- where higher values indicate a higher
confidence that a feature is a flare. Here we apopt a score of 0.5 as the
cutoff threshold, all features with a score of 0.5 or greater are classed as
flares and removed \citep[e.g.][]{FeinsteinFlare2020}.

Furthermore, we also bin the data from a 2-min to 10-min cadence using the
python package \texttt{lightkurve}'s binning function
\citep{LightkurveCollaborationLightkurve2018,GeertBarentsenKeplerGO2020}. This
further reduces any flaring-contribution that might have been missed by
\texttt{stella} \footnote{This is relevant for flares that are misshapen at the
start or break in the dataset due to missing either the ingress or egress.}
Subsequently, we filter photometry by only retaining data whos residules are
lless than 4 times the root-mean-square {\color{green}[Emily: Aylin, the RMS of
what?]} 

Gaussian processes for modeling the periods are based on
\citet{AngusInferring2018} for the subset of M Dwarfs with no fiducial periods.
The \texttt{starspot} \ package is adapted for light curve analysis
\citep{AngusRuthAngus2021} and accessible at
\url{https://zenodo.org/records/7697238}. Our Gaussian process kernel function
incorporates two stochastically-driven simple harmonic oscillators,
representing primary ($P_\textrm{rot}$) and secondary ($P_\textrm{rot}/2$)
rotation modes. First, we implement the Lomb-Scargle periodogram within
\texttt{starspot} to initially estimate the period. After which, we create a
maximum a posteriori (MAP) fit using \texttt{starspot} to generate a model for
stellar rotation. To obtain the posterior of the stellar rotation model, we use
Markov Chain Monte Carlo (MCMC) sampling using the \texttt{pymc3} package
\citep{SalvatierProbabilistic2016} within our adapted \texttt{starspot}
version. All rotational periods are presented in Table \ref{tab:dataset}. In
total we have access to 191 rotational periods.

\begin{deluxetable*}{cccccccc}
\tablehead{\colhead{ID} & \colhead{G Mag} & \colhead{V Mag} & \colhead{K Mag} & \colhead{R'$_{HK}$} & \colhead{R'$_{HK}$ err} & \colhead{Ro} & \colhead{prot}\\ \colhead{ } & \colhead{$\mathrm{mag}$} & \colhead{$\mathrm{mag}$} & \colhead{$\mathrm{mag}$} & \colhead{ } & \colhead{ } & \colhead{ } & \colhead{$\mathrm{d}$}}
\startdata
2MASS J00094508-4201396 & 12.140126 & 13.659000 & 8.223000 & -4.339205 & 0.001061 & 0.008610 & 0.859000 \\
2MASS J00310412-7201061 & 12.300688 & 13.648000 & 8.445000 & -5.387898 & 0.003143 & 0.928044 & 80.969000 \\
2MASS J01040695-6522272 & 12.447152 & 13.950000 & 8.532000 & -4.488843 & 0.001365 & 0.006320 & 0.624000 \\
2MASS J02004725-1021209 & 12.778251 & 14.113000 & 9.092000 & -4.790731 & 0.001247 & 0.188281 & 14.793000 \\
2MASS J02014384-1017295 & 13.026334 & 14.477000 & 9.189000 & -4.540044 & 0.001256 & 0.034402 & 3.152000 \\
2MASS J02125458+0000167 & 12.096099 & 13.580000 & 8.168000 & -4.634546 & 0.000999 & 0.048089 & 4.732000 \\
2MASS J02411510-0432177 & 12.250948 & 13.790000 & 8.246000 & -4.427184 & 0.001131 & 0.003768 & 0.400000 \\
2MASS J03100305-2341308 & 12.230226 & 13.500000 & 8.567000 & -4.233516 & 0.001114 & 0.027889 & 2.083000 \\
2MASS J03205178-6351524 & 12.086746 & 13.433000 & 8.195000 & -5.628891 & 0.004073 & 1.029200 & 91.622000 \\
2MASS J05015746-0656459 & 10.649305 & 12.196000 & 6.736000 & -5.004865 & 0.002015 & 0.874869 & 88.500000
\enddata

\input{tables/rotationalTableCaption.tex}
\end{deluxetable*}


One might expect a decrease in mean rotational period around the magnitude of
the gap, due to the slight decrease in magnetic activity. However, there is no
statisitically signifigant correlation between rotational period and G
magnitude which we can detect given our sample size (Figure
\ref{fig:rotationalSignifigance}). Rotational period is however, not the ideal
paraterization to use, as magnetic activity is more directly related to the
rossby number \citep[i.e.][]{}. Using the empiricial calibration presented in
\citet{Wright2018} (Equation \ref{eqn:tauc}) we find the mixing timescale for
each star and then let the rossby number, $Ro = P_{rot}/\tau_{c}$.

\begin{equation}
  \tau_{c} = 0.64 + 0.25 * (V-K)
\end{equation}

When we compare rossby number to G magnitude (Figure \ref{fig:rossby}) we find
that there ay be a slight paucity of rotation coincident with the decrease in
spread of the activity metric. We quantify the statistical signifigance of this
drop by building a gaussian kernel density estimator (kde) based on the data
outside of this range, and then resampling that kde 10000 times for each data
point in the theorized paucity range. The falsealarm probability that that drop
is due to noise is then the product of the fraction of samples which are less
than or equal to the value of each data point. We find that there is a 0.022
percent probabillity that this dip is due purley to noise.


\begin{figure}
  \centering
  \includegraphics[width=0.45\textwidth]{figures/RotationSignifigance.pdf}
  \caption{Rotational Periods against G magnitude for all stars with rotational
  periods (top). Standard deviation of rotational period within magnitude bin (bottom).}
  \label{fig:rotationalSignifigance}
\end{figure}

\begin{figure}
  \centering
  \includegraphics[width=0.45\textwidth]{figures/Rossby.pdf}
  \caption{Rossby number vs. G magnitude for all stars with rotational periods
  and V-K colors on Simbad. Dashed lines represent the hypothesized region of decreased rotation.}
  \label{fig:rossby}
\end{figure}


\subsection{Limitations}
There are two primary limitation of our dataset. First, we only have
{\color{red}232 stars} in our dataset limiting the statistical power of our
analysis. This is primarily due to the relative difficulty of obtaining Ca II
H\&K measurements compared to obtaining $H\alpha$ measurements. Reliable
measurements require both high spectral resolutions ({\color{red} R $\sim$
XXXXXX}) and a comparatively blue wavelength range \footnote{wrt. too what many
spectrographs cover. There is no unified resource listing currently
commissioned spectrographs; however, it is somewhat hard to source glass which
transmits well at H\&K wavelengths limiting the lower wavelength of most
spectrographs.}.

Additionally, the sample we do have does not extend to as low mass as would be
ideal. This presents a degeneracy between two potential causes for the observed
increased star-to-star variability. One option, as presented above and
elaborated on in the following section, is that this is due to kissing
instabilities. However, another possibility is that this increased variability
is intrinsic to the magnetic fields of fully convective stars. There is limited
discussion in the literature of the latter effect; however, \citet{Shulyak2019}
present estimated magnetic field strengths for 47 M dwarfs, spanning a larger
area around the convective transition region and their dataset does not
indicate a inherently increased variability for fully convective stars
({\color{red} fully confirm this, not just visually}).


\section{Modeling}\label{sec:modeling}
One of the most pressing questions related to this work is whether or not the
increased star-to-star variability in the activity metric and the Jao Gap,
which are coincident in magnitude, are driven by the same underlying mechanism.
The challenge when addressing this question arrises from current computational
limitations. Specifically, the kinds of three dimensional
magneto-hydrodynamical simulations --- which would be needed to derive the
effects of convective kissing instabilities on the magnetic field of the star
--- are infeasible to run over gigayear timescales while maintaing thermal
timescale resolutions needed to resolve periodic mixing events.

In order to address this and answer the specific question of \textit{could
kissing instabilities result in increased star-to-star variability of the
magnetic field}, we adopt a very simple toy model. Kissing instabilities result
in transient radiative zone seperating the core of a star (convective) from its
envelope (convective). When this radiative zone breaks down two important
things happen: one, the entire star becomes mechanically coupled, and two,
convective currents can now move over the entire radius of the star.
\citet{Jao2023} propose that this mecahnical coupling may allow the stars core
to act as an angular momentum sink thus accelerating a stars spin down and
resulting in anomolously low H$\alpha$ emission. 

Regardless of the exact mechanism by which the magnetic field may be effected,
it it reasonable to expect that both the mechanical coupling and the change to
the scale of convective currents will have some effect on the stars magnetic
field. On a microscopic scale both of these will change how packets of charge
within a star move and may serve to disrupt a stable dynamo. Therefore, in the
model we present here we make only one primary assumption: \textit{every mixing
event may modify the stars magnetic field by some amount}. Within our model
this assumption manifests as a random linear perturbation applied to some base
magnetic field at every mixing event. The strength of this perturbation is 
sampled from a normal distrubution with some standard deviation, $\sigma_{B}$.

Synthetic stars are sampled from a grid of stellar models evolved using the
Dartmouth Stellar Evolution Program (DSEP). Each stellar model was evolved
using a high temporal resolution (timesteps no larger than 10,000 years
{\color{red} Check this}) and typical numerical tolernances of one part in
$10^5$. Each model was based on a GS98 \citep{Grevesse1998} solar
composition with a mass range from 0.3 M$_{\odot}$ to 0.4 M$_{\odot}$. Finally,
models adopt OPLIB high temperature radiative opacities, Ferguson 2004 low
temperature radiative opacities, and include both atomic diffusion and
gravitational settling. A Kippenhan-Iben diagram showing the structural
evolution of a model within the gap is shown in Figure \ref{fig:kippenhan}.

\begin{figure*}
  \centering
  \includegraphics[width=0.9\textwidth]{figures/Kippenhan_clamped.pdf}
  \caption{Kippenhan-Iben diagram for a 0.345 solar mass star. Note the
  periodic mixing events (where the plotted curves peak).}
  \label{fig:kippenhan}
\end{figure*}

Each synthetic star is assigned some base magnetic activiy ($B_{0} \sim
\mathcal{N}(1, \sigma_{B})$) and then the number of mixing events before some age $t$
are counted based on local maxima in the core temperature. The toy magnetic
activiy at age $t$ for the model is given in Equation \ref{eqn:activity}. An
example of the magnetic evolution resulting from this model is given in Figure
\ref{fig:simpleB}. Fundamentally, this model presents magnetic
activity variation due to mixing events as a random walk and therefore results will
increasingly divergence over time.

\begin{align}\label{eqn:activity}
  B(t) = B_{0} + \sum_{i}B_{i} \sim \mathcal{N}(1, \sigma_{B}) 
\end{align}

\begin{figure}
  \centering
  \includegraphics[width=0.45\textwidth]{figures/simpleBEvolution.pdf}
  \caption{Example of the toy model presented here resulting in increased
  divergence between stars magnetic fields. The shaded region represents the
  maxium spread in the two point correlation function at each age.}
  \label{fig:simpleB}
\end{figure}

Applying the same analysis to these models as was done to the observations as
described in Section {\color{red} X.X} we find that this simple model results
in a qualitatively similar trend in the standard deviation vs. magnitude graph
(Figure \ref{fig:model}). The interpretation here is important, what this
qualitative similarity demonstrates is that it may be reasonable to expect
kissing instabilities to result in the observed increased star-to-star
variation. Importantly, we are not able to claim that kissing instabilities
\textit{do} lead to these increased variations, only that they reasonably
could. Further modeling, observational, and theoretical efforts will be needed
to more definitivley answer this question.

\begin{figure}
  \centering
  \includegraphics[width=0.45\textwidth]{figures/SpreadModel.pdf}
  \caption{Toy model results showing a qualitivativley similar discontinuoty in the star-to-star magnetic activity variability.}
  \label{fig:model}
\end{figure}

\subsection{Limitations}
The model presented in this paper is very limited and it is important to keep
those limitations in mind when interpreting the results presented here. Some of
the main challenges which should be leveled at this model are the assumption
that the magnetic field will be altered by some small random perturbation at
every ixing event. This assumption was informed by the large number of free
parameters avalible to a physical star during the establishment of a large scale 
magnetic field and the associated likeley stocastic nature of that process.
However, it is similarly belivable that the magnetic field will tend to alter in
a uniform manner at each mixing event. For example, since differnetial rotation
is generally proportinal to the temperature gradient within a star and activity is
strongly coupled to differential rotation then it may be that as the radiative zone reforms over thermal timescales the homoginization of angular momentum throughout the star results in overall lower amounts of differential rotation each after mixing event than would otherwise be present.

Moreover, this model does not consider how other degenerate sources of magnetic evolution such as stellar spin down, relaxation, or choronal heating may effect star-to-star variability. These could convably lead to a similar increase in star-to-star variability which is conincident with the Jao Gap magnitude as the switch from fully to partially convective may effect efficienty of these process.

Finally, we have not considered detailed descriptions of the dynamos of stars. The magnetohydrodynamical modeling which would be required to model the evolution of the magnetic field of these stars at thermal timescale resolutions over gigayears is currently beyond the ability of practical computing. Therefore future work should focus on limited modeling which may inform the evolution of the magnetic field directly around the time of a mixing event.


\input{sections/conclusion.tex}



\acknowledgments{
	This work has made use of the NASA astrophysical data system (ADS). We
	would like to thank Elisabeth Newton, Aaron Dotter, and Gregory Feiden for
	their support and for useful discussion related to the topic of this paper.
	We would like to thank our reviewer for their critical eye and their
	guidance to investigate to effects of the mixing length on the Gap
	Location. Additionally, we would like to thank James Colgan and the Los
	Alamos T-1 group for their assistance with the OPLIB opacity tables and
	support for the public release of \texttt{pyTOPSScrape}. We acknowledge the
	support of a NASA grant (No. 80NSSC18K0634). 
}
\acknowledgments

% \begin{acknowledgments}
% 	This work has made use of the NASA astrophysical data system (ADS). We
% 	would like to thank Elisabeth Newton, Aaron Dotter, and Gregory Feiden for
% 	their support and for useful discussion related to the topic of this paper.
% 	We would like to thank our reviewer for their critical eye and their
% 	guidance to investigate to effects of the mixing length on the Gap
% 	Location. Additionally, we would like to thank James Colgan and the Los
% 	Alamos T-1 group for their assistance with the OPLIB opacity tables and
% 	support for the public release of \texttt{pyTOPSScrape}. We acknowledge the
% 	support of a NASA grant (No. 80NSSC18K0634). 
% \end{acknowledgments}

\software{
	The Dartmouth Stellar Evolution Program (DSEP) \citep{Dotter2008},
	\texttt{BeautifulSoup} \citep{richardson2007beautiful},
	\texttt{mechanize} \citep{chandra2015python},
	\texttt{FreeEOS} \citep{Irwin2012},
	\texttt{pyTOPSScrape} \citep{Boudreaux22}
}


\bibliography{ms}{}
\bibliographystyle{aasjournal}


\end{document}

