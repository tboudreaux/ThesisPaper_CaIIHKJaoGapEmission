\documentclass[twocolumn]{aastex62}
\newcommand{\vdag}{(v)^\dagger}
\newcommand\aastex{AAS\TeX}
\newcommand\latex{La\TeX}

\usepackage{amsmath}
\usepackage{cancel}
\usepackage{float}

\shorttitle{Updated Opacities for DSEP}
\shortauthors{Boudreaux et al.}
% \watermark{DRAFT}
% \graphicspath{{./}{figures/}{src/figures}}

\begin{document}

\title{Correlations between Ca II H\&K Emission and the Gaia M dwarf Gap}

\correspondingauthor{Emily M. Boudreaux}
\email{thomas.m.boudreaux.gr@dartmouth.edu,\\thomas@boudreauxmail.com}

\author[0000-0002-2600-7513]{Emily M. Boudreaux}
\affiliation{Department of Physics and Astronomy, Dartmouth College, Hanover, NH 03755, USA}

\author[0000-0003-3096-4161]{Brian C. Chaboyer}
\affiliation{Department of Physics and Astronomy, Dartmouth College, Hanover, NH 03755, USA}

\received{}
\revised{}
\revised{}
\accepted{}
\submitjournal{ApJ}


\begin{abstract}
Previous work has demonstrated a paucity of Hydrogen Alpha emission at the same
  Gaia G Magnitude as the Jao Gap in the solar neighborhood. The exact
  mechanism which results in this paucity is as yet unknown; however, the
  authors of the originating paper suggestion that it may be the result of
  complex variations to a stars magnetic topology driven by the Jao Gaps
  characteristic formation and breakdown of stars radiative transition zones.
  Here I present a brief summary of a potential extension to this work looking
  at Ca II H\&K emission lines. Preliminary work with archival data shows a
  much stronger correlation between the calcium emission lines and the Jao Gap
  than was observed between hydrogen emission and the Jao Gap. If this
  observation withstands further testing then it may provide a new way to
  locate the gap in populations which lack the large counting statistics
  currently required --- and which are only practically available from Gaia
  data at moment.

\end{abstract}

\keywords{Stellar Evolution (1599) --- Stellar Evolutionary Models (2046)}

\section{INTRODUCTION}\label{sec:intro}
The initial mass requirements of molecular clouds collapsing to form stars
results in a strong bias towards lower masses and later spectral classes
during star formation. Partly as a result of this bias and partly as a result
of their extremely long main-sequence lifetimes, M Dwarfs make up approximately
70 percent of all stars in the galaxy \citep{Winters2019}. Moreover, many
planet search campaigns have focused on M Dwarfs due to the relative ease of
detecting small planets in their habitable zones \citep[e.g.][]{Nut08}. M
Dwarfs then represent both a key component of the galactic stellar population
as well as the most numerous possible set of stars which may host habitable
exoplanets. Given this key location M Dwarfs occupy in modern astronomy it is
important to have a thorough understanding of their structure and evolution.

\citet{Jao2018} discovered a novel feature in the Gaia Data Release 2 (DR2)
$G_{BP}-G_{RP}$ color-magnitude-diagram. Around $M_{G}=10$ there is an
approximately 17 percent decrease in stellar density of the sample of stars
\citet{Jao2018} considered. Subsequently, this has become known as either the
Jao Gap, or Gaia M Dwarf Gap. Following the initial detection of the Gap in DR2
the Gap has also potentially been observed in 2MASS \citep{Skrutskie2006,
Jao2018}; however, the significance of this detection is quite weak and it
relies on the prior of the Gap's location from Gaia data. The Gap is
also present in Gaia Early Data Release 3 (EDR3) \citep{Jao2021}. These EDR3
and 2MASS data sets then indicate that this feature is not a bias inherent to
DR2.

The Gap is generally attributed to convective instabilities in the cores of
stars straddling the fully convective transition mass (0.3 - 0.35 M$_{\odot}$)
\citep{Baraffe2018}. These instabilities interrupt the normal, slow, main
sequence luminosity evolution of a star and result in luminosities lower
than expected from the main sequence mass-luminosity relation \citep{Jao2020}.

The Jao Gap, inherently a feature of M Dwarf populations, provides an enticing
and unique view into the interior physics of these stars \citep{Feiden2021}.
This is especially important as, unlike more massive stars, M Dwarf seismology
is infeasible due to the short periods and extremely small
magnitudes which both radial and low-order low-degree non-radial seismic waves
are predicted to have in such low mass stars \citep{Rodriguez-Lopez2019}. The
Jao Gap therefore provides one of the only current methods to probe the
interior physics of M Dwarfs.

The magnetic activity of M dwarfs is of particular interest due to the
theorised links between habitability and the magnetic environment which a
planet resides within \citep[e.g.][]{Lammer2012,Gallet2017, Kislyakova2017}. M
dwarfs are known to be more magnetically active than earlier type stars
\citep{Saar1985,Astudillo-Defru2017,Wright2018} while simultaneously this same
high activity calls into question the canonical magnetic dynamo believed to
drive the magnetic field of solar-like stars (the $\alpha\Omega$ dynamo)
\citep{Shulyak2015}. One primary challenge which M dwarfs pose is that stars
less than approximately 0.35 M$_{\odot}$ are composed of a single convective
region. This denies any dynamo model differential rotation between adjacent
levels within the star. Alternative dynamo models have been proposed, such as
the $\alpha^{2}$ dynamo along with modifications to the $\alpha\Omega$ dynamo
which may be predictive of M dwarf magnetic fields \citep{Chabrier2006,
Kochukhov2021, Kleeorin2023}.

Despite this work, very few studies have dived specifically into the magnetic
field of M dwarfs at or near the convective transition region . This is not
surprising as that only spans approximately a 0.2 magnitude region
in the Gaia BP-RP color magnitude diagram and is therefore populated by a
relatively small sample of stars. 

\citet{Jao2023} identify the Jao Gap as a strong discontinuity point for
magnetic activity in M dwarfs. Two primary observations from their work are
that the Gap serves as a boundary where very few active stars, in their sample
of 640 M dwarfs, exist below the Gap and that the overall downward trend of
activity moving to fainter magnitudes is anomalously high in within the 0.2 mag
range of the Gap. \citeauthor{Jao2023} Figures 3 and 13 make this paucity in
H$\alpha$ emission particularly clear. Based on previous work from
\citet{Spada2020, Curtis2020, Dungee2022} the authors propose that the
mechanism resulting in the reduced fraction of active stars within the Gap is
that as the radiative zone dissipates due to core expansion, angular momentum
from the outer convective zone is dumped into the core resulting in a faster
spin down than would otherwise be possible. Effectively the core of the star
acts as a sink, reducing the amount of angular momentum which needs to be lost
by magnetic breaking for the outer convective region to reach the same angular
velocity. Given that H$\alpha$ emission is strongly coupled magnetic activity
in the upper chromosphere \citep{Newton2016, Kumar2023} and that a star's angular velocity
is a primary factor in its magnetic activity, a faster spin down will serve to
more quickly dampen H$\alpha$ activity.

In addition to H$\alpha$ the Calcium Fraunhaufer lines may be used to trace the
magnetic activity of a star. These lines originate from magnetic heating of the
lower chromosphere driven by magnetic shear stresses within the star.
Both \citet{Perdelwitz2021} and \citet{Boudreaux2022} present calcium emission
measurements for stars spanning the Jao Gap. In this paper we search for
similar trends in the Ca II H\& K emission as \citeauthor{Jao2023} see in the
H$\alpha$ emission. In Section \ref{sec:results} we investigate the empirical
star-to-star variability in emission and quantify if this could be due to noise
or sample bias; in Section \ref{sec:modeling} we present a simplified toy model
which shows that the mixing events characteristic of convective kissing
instabilities could lead to increased star-to-star variability in activity as
is seen empirically.

% Stellar modeling has been successful in reproducing the Jao Gap
% \citep[e.g.][]{Feiden2021,Mansfield2021} and, with these models, we have begun
% to understand which parameters constrain the Jao Gap's location. For example,
% it is now well documented that metallicity affects the Jao Gap's color, with
% higher metallicity stellar populations showing the Jao Gap at consistently
% higher masses / bluer colors \citep{Mansfield2021}.
%
% {\color{red} EXPAND THIS, READ SOME OTHER Gap PAPERS TO SEE WHAT THEY DO}

% Both \citeauthor{Feiden2021} and \citeauthor{Mansfield2021} demonstrate the Jao
% Gap's location sensitivity to age, evolving to higher mass regions of the
% mass-luminosity relation with population age. Per \citet{Mansfield2021} the
% degree of this location evolution also does not seem to be strongly sensitive
% to metallicity. 



\section{Conclusion}\label{sec:conclusion}
%
% \begin{figure}[H]
% 	\centering
% 	\includegraphics[width=0.45\textwidth]{SameMassConvectiveZoneComp.pdf}
% 	\caption{Portions of 0.3526 $M_{\odot}$ OPAL and OPLIB stellar models
% 	showing the interior shells which are radiative (black region). Note that
% 	for clarity only one convective mixing event from each model is shown. Note
% 	how the radiative zone in the OPLIB model is larger.}
% 	\label{fig:Unstable}
% \end{figure}
%
% \begin{figure}[H]
% 	\centering
% 	\includegraphics[width=0.45\textwidth]{Core3HECompSameMass.pdf}
% 	\caption{Core $^{3}$He mass fraction for  0.3526 $M_{\odot}$ models evolved
% 	with OPAL and OPLIB (within the Jao Gap's mass range for both). Note how
% 	the OPLIB model's core $^{3}$He mass fraction grows at approximately the
% 	same rate as the OPAL model's but continues uninterrupted for longer.}
% 	\label{fig:OPALOPLIB3He}
% \end{figure}
The Jao Gap provides an intriguing probe into the interior physics of M Dwarfs
stars where traditional methods of studying interiors break down. However,
before detailed physics may be inferred it is essential to have models which
are well matched to observations. Here we investigate whether the OPLIB opacity
tables reproduce the Jao Gap location and structure more accurately than the
widely used OPAL opacity tables. We find that while the OPLIB tables do shift
the Jao Gap location more in line with observations, by approximately 0.05
magnitudes, the shift is small enough that it is likely not distinguishable
from noise due to population age and chemical variation. However, future
measurement of [Fe/H] for stars within the gap will be helpful in constraining
the degree to which the gap should be smeared by these theoretical models.

We also find that both the color and magnitude of the Jao Gap are
correlated to the convective mixing length parameter. Specifically, a lower
mixing length parameter will bring the gap in the populations presented in this
paper more in line with the current best estimate for the actual gap magnitude.
Using this relation it may be possible for mixing length to be calibrated for
low mass stars such that models match the Jao Gap location. Further, the Jao
gap location may provide a test of alternative convection models such as
entropy calibrated convection \citep{Spada2021}. Both of these potential uses
require careful handeling of other uncertanties such as the uncertanties in
bolometric correction, popupulation composition, and population age. As we
currently do not have reason to suspect that the mixing length for the low mass
stars in the DR2 and ERD3 CMD is substantially lower than that of the sun we
leave the investigation of these potential additionl uses for future work.

Finally, we do not find that the OPLIB opacity tables help in reproducing the
as yet unexplained wedge shape of the observed Gap.




\acknowledgments{
	This work has made use of the NASA astrophysical data system (ADS). We
	would like to thank Elisabeth Newton, Aaron Dotter, and Gregory Feiden for
	their support and for useful discussion related to the topic of this paper.
	We would like to thank our reviewer for their critical eye and their
	guidance to investigate to effects of the mixing length on the Gap
	Location. Additionally, we would like to thank James Colgan and the Los
	Alamos T-1 group for their assistance with the OPLIB opacity tables and
	support for the public release of \texttt{pyTOPSScrape}. We acknowledge the
	support of a NASA grant (No. 80NSSC18K0634). 
}
\acknowledgments

% \begin{acknowledgments}
% 	This work has made use of the NASA astrophysical data system (ADS). We
% 	would like to thank Elisabeth Newton, Aaron Dotter, and Gregory Feiden for
% 	their support and for useful discussion related to the topic of this paper.
% 	We would like to thank our reviewer for their critical eye and their
% 	guidance to investigate to effects of the mixing length on the Gap
% 	Location. Additionally, we would like to thank James Colgan and the Los
% 	Alamos T-1 group for their assistance with the OPLIB opacity tables and
% 	support for the public release of \texttt{pyTOPSScrape}. We acknowledge the
% 	support of a NASA grant (No. 80NSSC18K0634). 
% \end{acknowledgments}

\software{
	The Dartmouth Stellar Evolution Program (DSEP) \citep{Dotter2008},
	\texttt{BeautifulSoup} \citep{richardson2007beautiful},
	\texttt{mechanize} \citep{chandra2015python},
	\texttt{FreeEOS} \citep{Irwin2012},
	\texttt{pyTOPSScrape} \citep{Boudreaux22}
}


\bibliography{ms}{}
\bibliographystyle{aasjournal}


\end{document}

